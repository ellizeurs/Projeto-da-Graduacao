\paragraph{}Este estudo investiga a dinâmica subjacente à série temporal semanal dos preços do biodiesel, influenciada majoritariamente pelo custo das matérias-primas e pela regulamentação vigente, que impacta diretamente a demanda.

\paragraph{} Com base na avaliação dos resultados, observa-se que, ao utilizar uma janela deslizante, a aplicação de técnicas de normalização e decomposição resultou em perda de consistência nos modelos neurais, exceto nos modelos morfológicos (\ac{IMP} e \ac{IDLN}), além de um aumento nos erros preditivos. Esses resultados sugerem que modelos baseados em decomposição linear não são adequados para essa série temporal. Ainda assim, o modelo \ac{N-Linear} demonstrou desempenho superior em comparação aos demais, tanto no conjunto de teste quanto no de validação, quando o treinamento foi realizado exclusivamente com a série de preços do biodiesel.

\paragraph{} Ao incluir a série de preços do óleo de soja como variável exógena, o modelo \ac{NHiTS} apresentou o melhor desempenho, como esperado devido à sua arquitetura profunda, que requer um maior volume de dados para treinamento. Contudo, dada sua complexidade, o modelo não mostrou uma vantagem substancial em relação ao \ac{MLP}. No conjunto de teste, o \ac{MLP} demonstrou alta precisão, enquanto o \ac{NHiTS} apresentou uma piora em relação ao desempenho observado na validação.

\paragraph{} Na análise dos modelos com janela de Takens, verificou-se novamente a perda de consistência ao aplicar a normalização quando o treinamento foi realizado somente com a série do biodiesel. Além disso, os valores da medida \(U_2\) excederam \(1\) em todos os casos, indicando que os modelos analisados tiveram um desempenho inferior ao de um passeio aleatório. Apesar disso, os modelos \ac{N-Linear} e \ac{MLP} obtiveram bons resultados no conjunto de teste, demonstrando a capacidade de lidar com conjuntos de dados de tamanho reduzido.

\paragraph{} Ao repetir a análise incluindo a série de preços do óleo de soja como covariável, observou-se uma melhora na consistência dos erros. Destacam-se, com baixos valores de erro, os modelos \ac{IDLN} e \ac{N-Linear}, ambos com valores reduzidos de \(U_2\). No conjunto de teste, ambos mantiveram o bom desempenho, mas o modelo \ac{MLP} merece destaque por superar os demais em todas as medidas de desempenho.

\paragraph{} A partir da análise realizada, observa-se que os modelos de redes neurais feed-forward e redes profundas demonstraram uma melhoria significativa com a inclusão de covariáveis, evidenciando sua capacidade de lidar com grandes volumes de dados. No entanto, nos testes, essa diferença foi observada em uma escala menor. As redes \ac{NARX} e \ac{DA-RNN} revelaram-se sensíveis ao conjunto de dados, exibindo variações consideráveis quando covariáveis foram adicionadas, em alguns casos apresentando melhores resultados com a série isolada. As redes morfológicas, por sua vez, mostraram eficiência com a janela deslizante, mas apresentaram desempenho inferior ao utilizar a janela de Takens.

\paragraph{} Entre os modelos avaliados, a rede \ac{N-Linear} destacou-se como a mais recomendada, pois apresentou melhor desempenho com a janela deslizante e manteve resultados consistentes ao utilizar a janela de Takens, evidenciando sua robustez em diferentes configurações de modelagem.

\paragraph{} Como trabalhos futuros, propõe-se o aprimoramento dos modelos com a inclusão de variáveis exógenas adicionais, como séries temporais de produção e demanda de biodiesel, preços de outras matérias-primas e o percentual obrigatório de mistura de biodiesel no diesel, para avaliar o impacto nos preços finais. Além disso, sugere-se a incorporação de modelos baseados em \ac{NLP} para análise do impacto de notícias no mercado. Recomenda-se também a aplicação de métodos de compactação de sinais, como \ac{PCA} e \ac{NLPCA}, para reduzir a dimensionalidade dos dados e analisar o comportamento dos modelos.

\paragraph{} Adicionalmente, propõe-se a inclusão de \acp{LSTM} menores, com o objetivo de reduzir a complexidade dos modelos e melhorar sua generalização. Sugere-se também a aplicação de métodos alternativos para extração de tendências em séries temporais, bem como a introdução de índices setoriais, como o índice agropecuário e o dólar, para uma análise mais abrangente do mercado. Além disso, destaca-se a importância da análise e ajuste dos modelos com base em histogramas e gráficos de dispersão, o que pode fornecer análises adicionais e aprimorar a qualidade das previsões.